\documentclass{article}

\usepackage{polski}


\begin{document}
	\section{Introduction}
	Ten artykuł przeznaczony jest do analizy sieci U-Net będącej jedną z wariantów Convolution Neural Network oraz jej wykorzystaniu w celu wykrywania dróg na zdjęciach satelitarnych. Sieć ta szczególnie znana jest w zastosowaniach medycznych.
	
U-Net is a specialized convolutional neural network (CNN) originally designed for image segmentation, particularly in biomedical imaging, where precise delineation of structures like organs, tumors, or cells is essential. It was proposed in Ronneberge et la.\cite{Ronneberger2015} Its name comes from the distinctive U-shaped architecture, which consists of two main pathways: a contracting path (encoder) that captures contextual information by progressively downsampling the image, and an expanding path (decoder) that reconstructs spatial details by upsampling back to the original image size. This symmetrical design enables the U-Net to combine both high-level semantic information and fine-grained localization in the final pixel-wise segmentation map.

The contracting path comprises a series of convolutional layers with activation functions (e.g., ReLU) and pooling operations that reduce spatial dimensions while increasing feature depth. This effectively extracts hierarchical features from the input. At the bottom of the “U” lies the bottleneck, which contains the most abstract representation of the input features. The expanding path then employs up-convolutions (also known as transposed convolutions) to restore spatial resolution. Importantly, it utilizes skip connections from corresponding levels of the encoder to preserve spatial information that might otherwise be lost during downsampling. These skip connections concatenate high-resolution encoder feature maps with decoder features, thus improving localization accuracy in the segmentation output.

Recent developments continue to extend and refine the original U-Net framework. A 2025 survey paper highlights how variants of U-Net incorporate mechanisms such as residual connections, transformer modules, and 3D adaptations to enhance segmentation performance across diverse medical imaging datasets. These advancements aim to improve the model’s ability to handle complex visual structures, incorporate broader contextual cues, and optimize performance even with limited annotated data—maintaining U-Net’s relevance as a foundational architecture in both medical and general image segmentation tasks.
	
	Trenowaliśmy model na dwóch różnych zbiorach danych. Pierwszy z nich składał się ze 100 zdjęć satelitarnych oraz 1000 sztucznie wygenerowanych obrazków \cite{kaggle.dataset.image_road}. Drugi zbrór to DeepGlobe 2018 Dataset \cite{Demir2018}.
	
	Za punkt wyjścia oraz benchmark wziąłem sieć U-net zaimplementowaną do tego zadania przez Tung Dinh \cite{github.unet}.
	
	
	\section{Our New Idea}
	Prpopozycją jest usprawnienie pierwotnej sieci przez modyfikację części warstw ukrytych.
	
\bibliography{references.bib}
\bibliographystyle{unsrt}
	
\end{document}
